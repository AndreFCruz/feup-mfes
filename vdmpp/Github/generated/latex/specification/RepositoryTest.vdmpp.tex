\begin{vdmpp}[breaklines=true]
class RepositoryTest
types
-- TODO Define types here
values
-- TODO Define values here
instance variables
 o : Account := new Organization("feup");
 r : Repository := new Repository("mfes", o, true);

operations
(*@
\label{testConstructor:11}
@*)
 private testConstructor: () ==> ()
 testConstructor() == (
  let org = new Organization("org"), repo = new Repository("test", org, true) in (
   TestUtils`assertTrue(repo.name = "test");
   TestUtils`assertTrue(repo.isRepoPrivate());
   TestUtils`assertTrue(card repo.collaborators = 0);
  );
  
  let usr = new User("usr"), repo = new Repository("test2", usr, false) in (
   TestUtils`assertTrue(card repo.collaborators = 1);
  );
  
  TestUtils`assertTrue(r.getDefaultBranch().name = "master");
  TestUtils`assertTrue(card dom r.branches = 1);
  TestUtils`assertTrue(r.branches("master") = r.getDefaultBranch());
 );

(*@
\label{testSetDescription:28}
@*)
 private testSetDescription: () ==> ()
 testSetDescription() == (
  r.setDescription(o, "Projeto de MFES");
  TestUtils`assertTrue(r.getDescription() = "Projeto de MFES");
  
  r.setDescription(o, "description");
  TestUtils`assertTrue(r.getDescription() = "description");
 );
 
(*@
\label{testAddCollaborator:37}
@*)
 private testAddCollaborator: () ==> ()
 testAddCollaborator() == (
  let u1 = new User("one"), u2 = new User("two") in (
   r.addCollaborator(u1);
   TestUtils`assertTrue(r.collaborators = {u1});
   
   r.addCollaborator(u2);
   TestUtils`assertTrue(r.collaborators = {u1, u2});
  );
 );
 
(*@
\label{testAddTag:48}
@*)
 private testAddTag: () ==> ()
 testAddTag() == (
  let tag1 = new Tag("AI"), tag2 = new Tag("WebDev") in (
   r.addTag(tag1);
   TestUtils`assertTrue(r.tags = {tag1});
   
   r.addTag(tag2);
   TestUtils`assertTrue(r.tags = {tag1, tag2});
  );
 );
 
(*@
\label{testCreateBranch:59}
@*)
 private testCreateBranch: () ==> ()
 testCreateBranch() == (
  let branch = r.createBranch("develop", true), b = r.branches("develop") in (
   TestUtils`assertTrue(branch.name = b.name);
   TestUtils`assertTrue(branch.name = "develop");
   
   TestUtils`assertTrue(branch.isProtected = b.isProtected);
   TestUtils`assertTrue(branch.isProtected);
  );
  TestUtils`assertTrue(card dom r.branches = 2);
 );
 
(*@
\label{testSetDefaultBranch:71}
@*)
 private testSetDefaultBranch: () ==> ()
 testSetDefaultBranch() == (
  TestUtils`assertTrue(r.getDefaultBranch().name = "master");
  r.setDefaultBranch("develop");
  TestUtils`assertTrue(r.getDefaultBranch().name = "develop");
 );
 
(*@
\label{testCommit:78}
@*)
 private testCommit: () ==> ()
 testCommit() == (
  let usr = new User("contributor"), pub = new Repository("public", o, false) in (
   r.addCollaborator(usr);
   TestUtils`assertTrue(len r.getDefaultBranch().getCommits() = 0);
   r.commit(usr, r.getDefaultBranch().name, "hash", new Date(2018, 12, 30, 22, 19));
   TestUtils`assertTrue(len r.getDefaultBranch().getCommits() = 1);
   
   -- Can also contribute to public repositories
   pub.commit(usr, pub.getDefaultBranch().name, "hash", new Date(2018, 12, 30, 22, 20));
   TestUtils`assertTrue(len pub.getDefaultBranch().getCommits() = 1);
  );
 );
 
(*@
\label{testCommitHistory:92}
@*)
 private testCommitHistory: () ==> ()
 testCommitHistory() == (
  TestUtils`assertTrue(len r.getDefaultBranch().getCommits() = 1);
  r.setDefaultBranch("master");
  TestUtils`assertTrue(len r.getDefaultBranch().getCommits() = 0);
 );
 
(*@
\label{testSetPrivacy:99}
@*)
 private testSetPrivacy: () ==> ()
 testSetPrivacy() == (
  TestUtils`assertTrue(r.isRepoPrivate());
  r.setPrivacy(o, false);
  TestUtils`assertFalse(r.isRepoPrivate());
 );
 
(*@
\label{testAddRelease:106}
@*)
 private testAddRelease: () ==> ()
  testAddRelease() ==
  (
   TestUtils`assertTrue(r.numReleases() = 0);
   r.addRelease(new Release("v1.1", new Date(2018, 12, 30, 22, 28)));
   TestUtils`assertTrue(r.numReleases() = 1);
   r.addRelease(new Release("v1.2", new Date(2018, 12, 30, 22, 29)));
   TestUtils`assertTrue(r.numReleases() = 2);
  );
 
(*@
\label{main:116}
@*)
 public static main: () ==> ()
 main() == (
  let rt = new RepositoryTest() in (
   rt.testConstructor();
   rt.testSetDescription();
   rt.testAddCollaborator();
   rt.testAddTag();
   rt.testAddRelease();
   rt.testCreateBranch();
   rt.testSetDefaultBranch();
   rt.testCommit();
   rt.testCommitHistory();
   rt.testSetPrivacy();
  );
 );

functions
-- TODO Define functiones here
traces
-- TODO Define Combinatorial Test Traces here
end RepositoryTest
\end{vdmpp}
\bigskip
\begin{longtable}{|l|r|r|r|}
\hline
Function or operation & Line & Coverage & Calls \\
\hline
\hline
\hyperref[main:116]{main} & 116&100.0\% & 1 \\
\hline
\hyperref[testAddCollaborator:37]{testAddCollaborator} & 37&100.0\% & 1 \\
\hline
\hyperref[testAddRelease:106]{testAddRelease} & 106&100.0\% & 1 \\
\hline
\hyperref[testAddTag:48]{testAddTag} & 48&100.0\% & 1 \\
\hline
\hyperref[testCommit:78]{testCommit} & 78&100.0\% & 1 \\
\hline
\hyperref[testCommitHistory:92]{testCommitHistory} & 92&100.0\% & 1 \\
\hline
\hyperref[testConstructor:11]{testConstructor} & 11&100.0\% & 1 \\
\hline
\hyperref[testCreateBranch:59]{testCreateBranch} & 59&100.0\% & 1 \\
\hline
\hyperref[testSetDefaultBranch:71]{testSetDefaultBranch} & 71&100.0\% & 1 \\
\hline
\hyperref[testSetDescription:28]{testSetDescription} & 28&100.0\% & 1 \\
\hline
\hyperref[testSetPrivacy:99]{testSetPrivacy} & 99&100.0\% & 1 \\
\hline
\hline
RepositoryTest.vdmpp & & 100.0\% & 11 \\
\hline
\end{longtable}

